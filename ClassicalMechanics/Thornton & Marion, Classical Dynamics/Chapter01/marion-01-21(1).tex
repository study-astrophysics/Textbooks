\documentclass[a4paper]{article}

\usepackage[paperwidth=21cm, paperheight=17cm, margin=0.2cm]{geometry}
\usepackage{amsmath}
\usepackage{amssymb}
\usepackage{amsfonts}
\usepackage{diagbox}
\usepackage{graphicx}
\usepackage[makeroom]{cancel}
\usepackage{xcolor}
\usepackage{verbatim}

\usepackage{titlesec}
\titleformat*{\section}{\Huge\bfseries}
\titlespacing*{\section}
{0pt}{5.5ex plus 1ex minus .2ex}{4.3ex plus .2ex}
\titleformat*{\subsection}{\LARGE\bfseries}
\titlespacing*{\subsection}
{0pt}{5.5ex plus 1ex minus .2ex}{4.3ex plus .2ex}


\begin{document}
\Large

From the Problem 1-11,

\begin{equation*}
	\mathbf{ABC} = (\vec{A} \times \vec{B}) \cdot \vec{C} = \left| \begin{array}{ccc} A_1 & A_2 & A_3 \\ B_1 & B_2 & B_3 \\ C_1 & C_2 & C_3 \end{array} \right|
\end{equation*}
\\
\begin{equation*}
	\mathbf{ABC} = A_1(B_2C_3-B_3C_2) - A_2(B_3C_1-B_1C_3) - A_3(B_2C_1-B_1C_2)
\end{equation*}
\\

From the definition of the \textbf{permutation symbol} (or \textbf{Levi-Civita density}) (Equation (1.67))

\begin{equation}
	\epsilon_{ijk} = 
	  \begin{cases}
	  	\ 0  & \text{if any index is equal to any other index} \\
	  	\ 1  & \text{if $i,j,k$ form and \textit{even} permutation of 1,2,3}\\
	  	-1 & \text{if $i,j,k$ form and \textit{odd} permutation of 1,2,3}
	  \end{cases},  \tag{1.67}
\end{equation}
\\

\begin{eqnarray*}
	\sum_{i,j,k} \epsilon_{ijk}A_iB_jC_k & = & A_1B_2C_3 - A_1B_3C_2 + A_2B_3C_1 - A_2B_1C_3 + A_3B_1C_2 - A_3B_2C_1 \\
	& = & A_1(B_2C_3-B_3C_2) - A_2(B_3C_1-B_1C_3) - A_3(B_2C_1-B_1C_2)
\end{eqnarray*}

Therefore,

\begin{equation*}
	\mathbf{ABC} = \sum_{i,j,k} \epsilon_{ijk}A_iB_jC_k
\end{equation*}

\end{document}