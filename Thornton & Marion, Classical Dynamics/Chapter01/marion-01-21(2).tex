\documentclass[a4paper]{article}

\usepackage[paperwidth=21cm, paperheight=16cm, margin=0.2cm]{geometry}
\usepackage{amsmath}
\usepackage{amssymb}
\usepackage{amsfonts}
\usepackage{diagbox}
\usepackage{graphicx}
\usepackage[makeroom]{cancel}
\usepackage{xcolor}
\usepackage{verbatim}

\usepackage{titlesec}
\titleformat*{\section}{\Huge\bfseries}
\titlespacing*{\section}
{0pt}{5.5ex plus 1ex minus .2ex}{4.3ex plus .2ex}
\titleformat*{\subsection}{\LARGE\bfseries}
\titlespacing*{\subsection}
{0pt}{5.5ex plus 1ex minus .2ex}{4.3ex plus .2ex}


\begin{document}
\Large

From the Equation 1.66 (the components of \textbf{vector product}), when $\vec{C} = \vec{A} \times \vec{B}$,

\begin{equation}
	C_i \equiv (\vec{A}\times\vec{B})_i = \sum_{j,k} \epsilon_{ijk}A_jB_k \tag{1.66}
\end{equation}
\\
and the definition of the \textbf{scalar product} (Equation 1.52),

\begin{equation}
  \vec{A}\cdot\vec{B} = \sum_i A_i B_i \tag{1.52}
\end{equation}
\\

Then, the \textbf{triple scalar product} is

\begin{eqnarray*}
	\mathbf{ABC} = \vec{A}\cdot(\vec{B}\times\vec{C}) & = & \sum_i A_i (\vec{B}\times\vec{C})_i \\
	& = & \sum_i A_i \Big(\sum_{j,k} \epsilon_{ijk}B_jC_k \Big) \\
	& = & \sum_i \sum_{j,k} \epsilon_{ijk}A_iB_jC_k \\
	& = & \sum_{i,j,k} \epsilon_{ijk}A_iB_jC_k
\end{eqnarray*}
Therefore,

\begin{equation*}
	\mathbf{ABC} = \sum_{i,j,k} \epsilon_{ijk}A_iB_jC_k
\end{equation*}

\end{document}