\documentclass[a4paper]{article}

\usepackage[paperwidth=21cm, paperheight=91cm, margin=0.2cm]{geometry}
\usepackage{amsmath}
\usepackage{amssymb}
\usepackage{amsfonts}
\usepackage{diagbox}
\usepackage{graphicx}
\usepackage[makeroom]{cancel}
\usepackage{xcolor}

\usepackage{titlesec}
\titleformat*{\section}{\Huge\bfseries}
\titlespacing*{\section}
{0pt}{5.5ex plus 1ex minus .2ex}{4.3ex plus .2ex}
\titlespacing*{\subsection}
{0pt}{5.5ex plus 1ex minus .2ex}{4.3ex plus .2ex}


\begin{document}
\Large
\section*{(a)  $ {\sum\limits_{i,j}} \epsilon_{ijk} \ \delta_{ij} = 0$}

From the definition of the \textbf{Kronecker delta symbol} (Equation (1.14))

\begin{equation}
	\delta_{ik} = 
	  \begin{cases}
	  	\ 0 & \text{if } i \neq k \\
	    \ 1 & \text{if } i = k
	  \end{cases}  \tag{1.14}
\end{equation}
\\
and \textbf{permutation symbol} (or \textbf{Levi-Civita density}) (Equation (1.67))
\begin{equation}
	\epsilon_{ijk} = 
	  \begin{cases}
	  	\ 0  & \text{if any index is equal to any other index} \\
	  	\ 1  & \text{if $i,j,k$ form and \textit{even} permutation of 1,2,3}\\
	  	-1 & \text{if $i,j,k$ form and \textit{odd} permutation of 1,2,3}
	  \end{cases},  \tag{1.67}
\end{equation}
\\

When $i = j$, 

\begin{eqnarray*}
	\delta_{ij} & = & 1 \\
	\epsilon_{ijk} & = & 0
\end{eqnarray*}

\begin{equation*}
	\therefore \ \sum_{ij} \delta_{ij} \ \epsilon_{ijk} = 0,
\end{equation*}
\\
and when $i \neq j$,

\begin{eqnarray*}
	\delta_{ij} & = & 0 \\
	\epsilon_{ijk} & = & 1
\end{eqnarray*}

\begin{equation*}
	\therefore \ \sum_{ij} \delta_{ij} \ \epsilon_{ijk} = 0.
\end{equation*}

Therfore,

\begin{equation*}
	\therefore \ \sum_{ij} \delta_{ij} \ \epsilon_{ijk} = 0.
\end{equation*}


\section*{(b)  $ {\sum\limits_{j,k}} \epsilon_{ijk} \ \epsilon_{ljk} = 2\delta_{il}$} 

\begin{eqnarray*}
	\sum_{j,k} \epsilon_{ijk} \ \epsilon_{ljk} & = & \cancel{\epsilon_{i11} \ \epsilon_{l11}} + \epsilon_{i12} \ \epsilon_{l12} + \epsilon_{i13} \ \epsilon_{l13} + \epsilon_{i21} \ \epsilon_{l21} + \cancel{\epsilon_{i22} \ \epsilon_{l22}} + \epsilon_{i23} \ \epsilon_{l23} \\
	&  & + \ \epsilon_{i31} \ \epsilon_{i31} + \epsilon_{i32} \ \epsilon_{l32} + \cancel{\epsilon_{i33} \ \epsilon_{l33}} \\\\
	& = & \epsilon_{i12} \ \epsilon_{l12} + \epsilon_{i13} \ \epsilon_{l13} + \epsilon_{i21} \ \epsilon_{l21} + \epsilon_{i23} \ \epsilon_{l23} + \epsilon_{i31} \ \epsilon_{i31} + \epsilon_{i32} \ \epsilon_{l32}
\end{eqnarray*}

If $i = l$,

\begin{eqnarray*}
	\sum_{j,k} \epsilon_{ijk} \ \epsilon_{ljk} & = & (\epsilon_{i12})^2 + (\epsilon_{i13})^2 + (\epsilon_{i21})^2 + (\epsilon_{i23})^2 + (\epsilon_{i31})^2 + (\epsilon_{i32})^2 \\
	& = & 0^2 + 0^2 + 0^2 + 1^2 + 0^2 + (-1)^2 \quad = 2 \text{\qquad if $i = 1$} \\\\
	& = & 0^2 + (-1)^2 + 0^2 + 0^2 + 1^2 + 0^2 \quad = 2 \text{\qquad if $i = 2$} \\\\
	& = & 1^2 + 0^2 + (-1)^2 + 0^2 + 0^2 + 0^2 \quad = 2 \text{\qquad if $i = 3$}.
\end{eqnarray*}

\begin{equation*}
	\therefore \ \sum_{j,k} \epsilon_{ijk} \ \epsilon_{ljk} = 2 \text{\qquad when $i = l$}
\end{equation*}
And if $i \neq l$,

\begin{eqnarray*}
	\sum_{j,k} \epsilon_{ijk} \ \epsilon_{ljk} & = & \epsilon_{i12} \ \epsilon_{l12} + \epsilon_{i13} \ \epsilon_{l13} + \epsilon_{i21} \ \epsilon_{l21} + \epsilon_{i23} \ \epsilon_{l23} + \epsilon_{i31} \ \epsilon_{i31} + \epsilon_{i32} \ \epsilon_{l32}\\
	& = & 0\cdot0 + 0\cdot(-1) + 0\cdot0 + 1\cdot0 + 0\cdot1 + (-1)\cdot0 \quad = 0 \text{\qquad when $i=1,\ l=2$} \\\\
	& = & 0\cdot(-1) + 0\cdot0 + 0\cdot(-1) + 1\cdot0 + 0\cdot0 + (-1)\cdot0 \quad = 0 \text{\qquad when $i=1, \ l=3$} \\\\
	& = & 0\cdot0 + (-1)\cdot0 + 0\cdot0 + 0\cdot1 + 1\cdot0 + 0\cdot(-1) \quad = 0 \text{\qquad when $i=2,\ l=1$} \\\\
	& = & 0\cdot1 + (-1)\cdot0 + 0\cdot(-1) + 0\cdot0 + 1\cdot0 + 1\cdot0 \quad = 0 \text{\qquad when $i=2,\ l=3$} \\\\
	& = & (-1)\cdot0 + 0\cdot0 + (-1)\cdot0 + 0\cdot1 + 0\cdot0 + 0\cdot(-1) \quad = 0 \text{\qquad when $i=3,\ l=1$} \\\\
	& = & (-1)\cdot0 + 0\cdot(-1) + (-1)\cdot0 + 0\cdot0 + 0\cdot1 + 0\cdot0 \quad = 0 \text{\qquad when $i=3,\ l=2$}
\end{eqnarray*}

\begin{equation*}
	\therefore \ \sum_{j,k} \epsilon_{ijk} \ \epsilon_{ljk} = 0 \text{\qquad when $i \neq l$}
\end{equation*}
\\

Therefore,

\begin{equation*}
	\sum_{j,k} \epsilon_{ijk} \ \epsilon_{ljk} = 2 \delta_{il} = 
	\begin{cases}
	  	\ 0 & \text{if } i \neq l \\
	    \ 2 & \text{if } i = l
	  \end{cases}
\end{equation*}


\section*{(c) $ {\sum\limits_{i,j,k}} \epsilon_{ijk} \ \epsilon_{ijk} = 6$}

When $i = 1$, then the $\epsilon_{ijk}$ is \\

\begin{center}
\scalebox{2.0}{
\begin{tabular}{|c|c|c|c|}
\hline
\diagbox{j}{k} & 1 & 2 & 3 \\
\hline
1 & 0 & 0 & 0 \\ \hline
2 & 0 & 0 & 1 \\\hline
3 & 0 & -1& 0 \\
\hline
\end{tabular}	
}
\vspace*{1 cm}
\end{center}


When $i = 2$, then the $\epsilon_{ijk}$ is \\

\begin{center}
\scalebox{2.0}{
\begin{tabular}{|c|c|c|c|}
\hline
\diagbox{j}{k} & 1 & 2 & 3 \\
\hline
1 & 0 & 0 &-1 \\ \hline
2 & 0 & 0 & 0 \\\hline
3 & 1 & 0 & 0 \\
\hline
\end{tabular}	
}
\vspace*{1 cm}
\end{center}


When $i = 3$, then the $\epsilon_{ijk}$ is \\

\begin{center}
\scalebox{2.0}{
\begin{tabular}{|c|c|c|c|}
\hline
\diagbox{j}{k} & 1 & 2 & 3 \\
\hline
1 & 0 & 1 & 0 \\ \hline
2 &-1 & 0 & 0 \\\hline
3 & 0 & 0 & 0 \\
\hline
\end{tabular}	
}
\vspace*{1 cm}
\end{center}

Therefore,

\begin{equation*}
  \sum_{i,j,k} \epsilon_{ijk} \ \epsilon_{ijk} = 1^2 + (-1)^2 + (-1)^2 + 1^2 + 1^2 + (-1)^2 = 6
\end{equation*}


\end{document}