\documentclass[a4paper]{article}

\usepackage[paperwidth=21cm, paperheight=16cm, margin=0.2cm]{geometry}
\usepackage{amsmath}
\usepackage{amssymb}
\usepackage{amsfonts}
\usepackage{diagbox}
\usepackage{graphicx}
\usepackage[makeroom]{cancel}
\usepackage{xcolor}
\usepackage{verbatim}

\usepackage{titlesec}
\titleformat*{\section}{\Huge\bfseries}
\titlespacing*{\section}
{0pt}{5.5ex plus 1ex minus .2ex}{4.3ex plus .2ex}
\titleformat*{\subsection}{\LARGE\bfseries}
\titlespacing*{\subsection}
{0pt}{5.5ex plus 1ex minus .2ex}{4.3ex plus .2ex}


\begin{document}
\Large

\section{Kronecker delta Symbol}

From the definition of the \textbf{Kronecker delta symbol} (Equation (1.14))

\begin{equation}
	\delta_{ik} = 
	  \begin{cases}
	  	\ 0 & \text{if } i \neq k \\
	    \ 1 & \text{if } i = k
	  \end{cases}  \tag{1.14}
\end{equation}

\section{Levi-Civita Symbol}

and \textbf{permutation symbol} (or \textbf{Levi-Civita density}) (Equation (1.67))
\begin{equation}
	\epsilon_{ijk} = 
	  \begin{cases}
	  	\ 0  & \text{if any index is equal to any other index} \\
	  	\ 1  & \text{if $i,j,k$ form and \textit{even} permutation of 1,2,3}\\
	  	-1 & \text{if $i,j,k$ form and \textit{odd} permutation of 1,2,3}
	  \end{cases},  \tag{1.67}
\end{equation}


\section{Pythagorean identity}

\begin{equation}
  \sin^2(\alpha-\beta)+\cos^2(\alpha-\beta) = 1
\end{equation}


\section{If you want to put a cancel mark}

\begin{comment}
\cancel{}
\end{comment}

\begin{eqnarray*}
	\sum_{j,k} \epsilon_{ijk} \ \epsilon_{ljk} & = & \cancel{\epsilon_{i11} \ \epsilon_{l11}} + \epsilon_{i12} \ \epsilon_{l12} + \epsilon_{i13} \ \epsilon_{l13} + \epsilon_{i21} \ \epsilon_{l21} + \cancel{\epsilon_{i22} \ \epsilon_{l22}} + \epsilon_{i23} \ \epsilon_{l23} \\
	&  & + \ \epsilon_{i31} \ \epsilon_{i31} + \epsilon_{i32} \ \epsilon_{l32} + \cancel{\epsilon_{i33} \ \epsilon_{l33}} \\\\
	& = & \epsilon_{i12} \ \epsilon_{l12} + \epsilon_{i13} \ \epsilon_{l13} + \epsilon_{i21} \ \epsilon_{l21} + \epsilon_{i23} \ \epsilon_{l23} + \epsilon_{i31} \ \epsilon_{i31} + \epsilon_{i32} \ \epsilon_{l32}
\end{eqnarray*}

\section{Scalar Product}

and the definition of the \textbf{scalar product} (Equation 1.52),

\begin{equation}
  \vec{A}\cdot\vec{B} = \sum_i A_i B_i \tag{1.52}
\end{equation}


\section{Vector Product}

From the Equation 1.66 (the components of \textbf{vector product}), when $\vec{C} = \vec{A} \times \vec{B}$,

\begin{equation}
	C_i \equiv (\vec{A}\times\vec{B})_i = \sum_{j,k} \epsilon_{ijk}A_jB_k \tag{1.66}
\end{equation}

\end{document}